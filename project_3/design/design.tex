\documentclass[UTF8]{ctexart}
\usepackage{amsmath}
\usepackage{float}
\usepackage{hyperref}
\usepackage{indentfirst}
\usepackage{cite}
\usepackage{multicol}
\usepackage{geometry}
\usepackage{txfonts}
\usepackage{extarrows}
\usepackage{graphicx}
\usepackage{booktabs}
\usepackage{diagbox}
\usepackage{threeparttable}
\usepackage{caption}
\usepackage[usenames,dvipsnames]{color}
\usepackage{colortbl}
\usepackage{fontspec}
\usepackage{listings}
\usepackage{mips}
\usepackage{fancybox}
\usepackage[super,square]{natbib}

\makeatletter
\newenvironment{CenteredBox}{% 
\begin{Sbox}}{% Save the content in a box
\end{Sbox}\centerline{\parbox{\wd\@Sbox}{\TheSbox}}}% And output it centered
\makeatother


\definecolor{dkgreen}{rgb}{0,0.6,0}
\definecolor{gray}{rgb}{0.5,0.5,0.5}
\definecolor{mauve}{rgb}{0.58,0,0.82}
\newfontfamily\courier{Courier New}
\lstset{ %
  language=[mips]Assembler,       % the language of the code
  numbers=left,
  numberstyle=\small\courier,
  basicstyle=\small\courier
}
\captionsetup[table]{singlelinecheck = false,justification=raggedleft}
\setmainfont{Times New Roman}
\definecolor{mypink}{rgb}{.99,.91,.95}
\geometry{left=0.3cm,right=2cm,top=2.5cm,bottom=2cm}
\setlength{\parindent}{2em}
\addtolength{\parskip}{-.4em}
\title{\textbf{P3设计文档}}
\author{伦泽标}
\date{\today}
\begin{document}
\pagestyle{plain}

\null
\nointerlineskip
\vfill
\let\snewpage \newpage
\let\newpage \relax
\maketitle
\let \newpage \snewpage
\vfill
\break
\section{Instruction Set}
\begin{table}[H]
	\begin{threeparttable}
		\begin{tabular}{cccl}
			\toprule
			\rowcolor{mypink}
			\textbf{Inst} & \textbf{Part}                           & \textbf{Value}                & \textbf{Comment}                                                \\
			\midrule
			addu&{Op[6],rs[5],rt[5],rd[5],00000,Funct[6]}&Op=6'b000000, Funct=6b'100001\\
			\midrule
			subu&{Op[6],rs[5],rt[5],rd[5],00000,Funct[6]}&Op=6'b000000, Funct=6b'100011\\
			\midrule
			ori           & {Op[6],rs[5],rt[5],imm[16]}             & Op=6'b001101                  & rt$\leftarrow$rs | zero\_extend(imm)                            \\
			\midrule
			lw            & {Op[6],base[5],rt[5],offset[16]}        & Op=6'b100011                  & rt$\leftarrow$memory[base+offset], sign\_extend(offset)         \\
			\midrule
			sw            & {Op[6],base[5],rt[5],offset[16]}        & Op=6'b101011                  & memory[base+offset]$\leftarrow$rt, sign\_extend(offset)         \\
			\midrule
			beq           & {Op[6],rs[5],rt[5],offset[16]}          & Op=6'b000100                  & sign\_extend(offset)<<2                                         \\
			\midrule
			lui           & {Op[6],00000,rt[5],imm[16]}             & Op=6'b001111                  & rt$\leftarrow$imm<<16                                           \\
			\midrule
			sll           & {Op[6],00000,rt[5],rd[5],s[5],Funct[6]} & Op=6'b000000, Funct=6'b000000 & rd$\leftarrow$rt<<s                                             \\
			\midrule
			sllv          & {Op[6],rs[5],rt[5],rd[5],0*5,Funct}     & Op=6'b000000, Funct=6'b000100 & rd$\leftarrow$rt<<rs[4:0]                                       \\
			\midrule
			sra           & {Op[6],0*5,rt[5],rd[5],s[5],Funct}      & Op=6'b000000, Funct=6'b000011 & rd$\leftarrow$rt>>s,arithmetic                                  \\
			\midrule
			srav          & {Op[6],rs[5],rt[5],rd[5],0*5,Funct[6]}  & Op=6'b000000, Funct=6'b000111 & rd$\leftarrow$rt>>rs[4:0],arithmetic                           \\
			\midrule
			srl           & {Op[6],0*5,rt[5],rd[5],s[5],Funct[6]}   & Op=6'b000000, Funct=6'b000010 & rd$\leftarrow$rt>>s,logic                                      \\
			\midrule
			srlv          & {Op[6],rs[5],rt[5],rd[5],0*5,Funct[6]}  & Op=6'b000000, Funct=6'b000110 & rd$\leftarrow$rt>>rs[4:0],logic                                \\
			\midrule
			j             & {Op[6], j\_addr[26]}                    & Op=6'b000010                  & PC $\leftarrow$ \{PC[31:28],j\_addr,00\}                        \\
			\midrule
			jr            & {Op[6],rs[5],0*10,0*5,Funct[6]}         & Op=6'b000000, Funct=6'b001000 & PC$\leftarrow$rs                                                \\
			\midrule
			jal           & {Op[6],j\_addr[26]}                     & Op=6'b000011                  & PC $\leftarrow$ \{PC[31:28],j\_addr,00\}, \$ra$\leftarrow$ PC+4 \\
			\midrule
			jalr          & {Op[6],rs[5],0*5,rd[5],0*5,Funct[6]}    & Op=6'b000000, Funct=6'b001001 & PC$\leftarrow$ rs,rd$\leftarrow$PC+4                            \\
			\midrule
			nop           & 0                                       & 0                             & sll \$0,\$0,0                                                   \\
			\midrule
		\end{tabular}
	\end{threeparttable}
\end{table}
\newpage
\section{Overview}
CPU顶层结构如下图所示
\begin{figure}[H]
	\includegraphics[scale=0.5]{CPU.png}
	\caption{CPU顶层结构}
\end{figure}
\newpage
\section{Unit}
\subsection{GRF}
\begin{table}[H]
	\centering
	\begin{threeparttable}
		\caption{GRF端口定义}
		\begin{tabular}{cccc}
			\toprule
			\rowcolor{mypink}
			\textbf{Port} & \textbf{Direction} & \textbf{Width} & \textbf{Description} \\
			\midrule
			Clk           & Input              & 1              & 时钟信号             \\
			\midrule
			Reset         & Input              & 1              & 复位信号             \\
			\midrule
			WE            & Input              & 1              & 写使能信号           \\
			\midrule
			A1            & Input              & [4:0]          & 读寄存器地址1        \\
			\midrule
			A2            & Input              & [4:0]          & 读寄存器地址2        \\
			\midrule
			A3            & Input              & [4:0]          & 写寄存器地址         \\
			\midrule
			Data          & Input              & [31:0]         & 数据输入             \\
			\midrule
			RD1           & Output             & [31:0]         & A1所指向寄存器的值   \\
			\midrule
			RD2           & Output             & [31:0]         & A2所指向寄存器的值   \\
			\midrule
		\end{tabular}
	\end{threeparttable}
\end{table}
\begin{table}[H]
	\centering
	\begin{threeparttable}
		\caption{GRF功能定义}
		\begin{tabular}{cc}
			\toprule
			\rowcolor{mypink}
			\textbf{Port} & \textbf{Function}                                       \\
			\midrule
			Reset         & 异步复位                                                \\
			\midrule
			WE            & 当Clk上升沿来临,若WE为高电平,则将Data写入A3指向的地址 \\
			\midrule
		\end{tabular}
	\end{threeparttable}
\end{table}
\begin{figure}[H]
	\centering
	\includegraphics[scale=0.5]{GRF.png}
	\caption{GRF实现图}
\end{figure}
\newpage
\subsection{ALU}
\begin{table}[H]
	\centering
	\begin{threeparttable}
		\caption{ALU端口定义}
		\begin{tabular}{cccc}
			\toprule
			\rowcolor{mypink}
			\textbf{Port} & \textbf{Direction} & \textbf{Width} & \textbf{Description} \\
			\midrule
			SrcA          & Input              & [31:0]         & ALU输入1             \\
			\midrule
			SrcB          & Input              & [31:0]         & ALU输入2             \\
			\midrule
			ALUCtrl       & Input              & [3:0]          & ALU功能控制          \\
			\midrule
			Zero          & Output             & 1              & ALU计算结果是否为零  \\
			\midrule
			ALURes        & Output             & [31:0]         & ALU计算结果          \\
			\midrule
		\end{tabular}
	\end{threeparttable}
\end{table}
\begin{table}[H]
	\centering
	\begin{threeparttable}
		\caption{ALU功能定义}
		\begin{tabular}{cc}
			\toprule
			\rowcolor{mypink}
			\textbf{ALUCtrl} & \textbf{Function}          \\
			\midrule
			4'b0000          & 加法运算                   \\
			\midrule
			4'b0001          & 减法运算                   \\
			\midrule
			4'b0010          & 与运算                     \\
			\midrule
			4'b0011          & 或运算                     \\
			\midrule
			4'b0100          & 左移运算(SrcB<<SrcA[4:0])  \\
			\midrule
			4'b0101          & 逻辑右移(SrcB>>SrcA[4:0])  \\
			\midrule
			4'b0110          & 算术右移(SrcB>>>SrcA[4:0]) \\
			\midrule
			4'b0111          & 加载到高位                 \\
			\midrule
			4'b1000          & 比较大小(SrcA<SrcB)无符号  \\
			\midrule
			4'b1001          & 比较大小(SrcA>SrcB)有符号  \\
			\midrule
		\end{tabular}
	\end{threeparttable}
\end{table}
\begin{figure}[H]
	\centering
	\includegraphics[scale=0.5]{ALU.png}
	\caption{ALU实现图}
\end{figure}
\newpage
\subsection{EXT}
\begin{table}[H]
	\centering
	\begin{threeparttable}
		\caption{EXT端口定义}
		\begin{tabular}{cccc}
			\toprule
			\rowcolor{mypink}
			\textbf{Port} & \textbf{Direction} & \textbf{Width} & \textbf{Description}     \\
			\midrule
			Imm           & Input              & [15:0]         & 输入                     \\
			\midrule
			ExtOp         & Input              & [2:0]          & Extender功能控制(有冗余) \\
			\midrule
			ExtRes        & Output             & [31:0]         & 扩展结果                 \\
			\midrule
		\end{tabular}
	\end{threeparttable}
\end{table}
\begin{table}[H]
	\centering
	\begin{threeparttable}
		\caption{EXT功能定义}
		\begin{tabular}{cc}
			\toprule
			\rowcolor{mypink}
			\textbf{ExtOp} & \textbf{Function} \\
			\midrule
			3'b000         & 零扩展            \\
			\midrule
			3'b001         & 符号扩展          \\
			\midrule
			3'b010         & 加载到高位        \\
			\midrule
		\end{tabular}
	\end{threeparttable}
\end{table}
\begin{figure}[H]
	\centering
	\includegraphics[scale=0.7]{Extender.png}
	\caption{EXT实现图}
\end{figure}
\newpage
\subsection{DM}
\begin{table}[H]
	\centering
	\begin{threeparttable}
		\caption{DM端口定义}
		\begin{tabular}{cccc}
			\toprule
			\rowcolor{mypink}
			\textbf{Port} & \textbf{Direction} & \textbf{Width} & \textbf{Description}       \\
			\midrule
			Clk           & Input              & 1              & 时钟信号                   \\
			\midrule
			Reset         & Input              & 1              & 复位信号                   \\
			\midrule
			Addr          & Input              & [31:0]         & 目标存储器地址(仅取后五位) \\

			\midrule
			Data          & Input              & [31:0]         & 数据输入                   \\
			\midrule
			WE            & Input              & 1              & 写入使能信号               \\
			\midrule
			RD            & Output             & [31:0]         & 读取数据                   \\
			\midrule
		\end{tabular}
	\end{threeparttable}
\end{table}
\begin{table}[H]
	\centering
	\begin{threeparttable}
		\caption{DM功能定义}
		\begin{tabular}{cc}
			\toprule
			\rowcolor{mypink}
			\textbf{Port} & \textbf{Function}                                         \\
			\midrule
			Reset         & 异步复位                                                  \\
			\midrule
			WE            & 当Clk上升沿来临,若WE为高电平,则将Data写入Addr指向的地址 \\
			\midrule
		\end{tabular}
	\end{threeparttable}
\end{table}
\begin{figure}[H]
	\centering
	\includegraphics[scale=0.8]{DM.png}
	\caption{DM实现图}
\end{figure}
\newpage
\subsection{IFU}
\begin{table}[H]
	\centering
	\begin{threeparttable}
		\caption{IFU端口定义}
		\begin{tabular}{cccc}
			\toprule
			\rowcolor{mypink}
			\textbf{Port} & \textbf{Direction} & \textbf{Width} & \textbf{Description}            \\
			\midrule
			Clk           & Input              & 1              & 时钟信号                        \\
			\midrule
			Reset         & Input              & 1              & 复位信号                        \\
			\midrule
			PCSrc         & Input              & 1              & 来自Branch模块,是否b跳转       \\
			\midrule
			Imm           & Input              & [16:0]         & 用于计算跳转偏移量              \\
			\midrule
			Jump          & Input              & 1              & 来自Controller,是否j跳转       \\
			\midrule
			J\_Addr       & Input              & [25:0]         & j跳转偏移量                     \\
			\midrule
			RD            & Input              & [31:0]         & 寄存器rs的值(jr,jalr)           \\
			\midrule
			Jump\_R       & Input              & 1              & 来自Controller,是否jr/jalr跳转 \\
			\midrule
			PC+4          & Output             & [31:0]         & 当前PC+4,用于Link              \\
			\midrule
			Inst          & Output             & [31:0]         & 当前指令                        \\
			\midrule
		\end{tabular}
	\end{threeparttable}
\end{table}
\begin{table}[H]
	\centering
	\begin{threeparttable}
		\caption{IFU功能定义}
		\begin{tabular}{cc}
			\toprule
			\rowcolor{mypink}
			\textbf{Port} & \textbf{Function}                                  \\
			\midrule
			Clk           & 上升沿更新指令                                     \\
			\midrule
			Reset         & 异步复位                                           \\
			\midrule
			PCSrc         & 低电平表示下一条指令为PC+4,高电平为PC+4+SigImm<<2 \\
			\midrule
			Jump          & 高电平表示下一条指令为\{PC[31:28],J\_Addr,00\}     \\
			\midrule
		\end{tabular}
	\end{threeparttable}
\end{table}
\begin{figure}[H]
	\centering
	\includegraphics[scale=0.6]{IFU.png}
	\caption{IFU实现图}
\end{figure}
\newpage
\subsection{Branch(用于控制跳转)}
\begin{table}[H]
	\centering
	\begin{threeparttable}
		\caption{Branch端口定义}
		\begin{tabular}{cccc}
			\toprule
			\rowcolor{mypink}
			\textbf{Port} & \textbf{Direction} & \textbf{Width} & \textbf{Description} \\
			\midrule
			Branch        & Input              & 1              & 是否进行b跳转判定    \\
			\midrule
			BranchOp      & Input              & [3:0]          & b跳转类型            \\
			\midrule
			SrcA          & Input              & [31:0]         & 操作数1              \\
			\midrule
			SrcB          & Input              & [31:0]         & 操作数2              \\
			\midrule
			PCSrc         & Output             & 1              & 是否b跳转            \\
			\midrule
		\end{tabular}
	\end{threeparttable}
\end{table}
\begin{table}[H]
	\centering
	\begin{threeparttable}
		\caption{Branch功能定义}
		\begin{tabular}{cc}
			\toprule
			\rowcolor{mypink}
			\textbf{BranchOp} & \textbf{Function} \\
			\midrule
			4'b0000           & beq               \\
			\midrule
			4'b0001           & bne               \\
			\midrule
			4'b0010           & bgez              \\
			\midrule
			4'b0011           & bgtz              \\
			\midrule
			4'b0100           & blez              \\
			\midrule
			4'b0101           & bltz              \\
			\midrule
		\end{tabular}
	\end{threeparttable}
\end{table}
\begin{figure}[H]
	\centering
	\includegraphics[scale=0.8]{Branch.png}
	\caption{Branch实现图}
\end{figure}
\newpage
\section{Controller}
\begin{table}[H]
	\centering
	\begin{threeparttable}
		\caption{Controller端口定义}
		\begin{tabular}{cccc}
			\toprule
			\rowcolor{mypink}
			\textbf{Port} & \textbf{Direction} & \textbf{Width} & \textbf{Description}            \\
			\midrule
			Op            & Input              & [5:0]          & OpCode                          \\
			\midrule
			Funct         & Input              & [5:0]          & Funct                           \\
			\midrule
			MemtoReg      & Output             & 1              & 从DM读取数据到GRF               \\
			\midrule
			MemWrite      & Output             & 1              & 向DM写入来自GRF的数据           \\
			\midrule
			Branch        & Output             & 1              & b跳转                           \\
			\midrule
			BranchOp      & Output             & [3:0]          & b跳转类型                       \\
			\midrule
			ALUCtrl       & Output             & [3:0]          & ALU功能控制                     \\
			\midrule
			ALUASrc       & Output             & 1              & ALU.SrcA来源选择                \\
			\midrule
			ALUSrc        & Output             & 1              & ALU.SrcB来源选择                \\
			\midrule
			RegDst        & Output             & 1              & 目标寄存器选择([20:16],[15:11]) \\
			\midrule
			RegWrite      & Output             & 1              & 寄存器写使能                    \\
			\midrule
			Extend        & Output             & 1              & 扩展                            \\
			\midrule
			Jump          & Output             & 1              & j跳转                           \\
			\midrule
			Jump\_R       & Output             & 1              & j跳转(r)                        \\
			\midrule
			Link          & Output             & 1              & j跳转(l)                        \\
			\midrule
		\end{tabular}
	\end{threeparttable}
\end{table}
\begin{table}[H]
	\centering
	\begin{threeparttable}
		\caption{OpCode译码}
		\begin{tabular}{cc}
			\toprule
			\rowcolor{mypink}
			\textbf{Op} & \textbf{Inst} \\
			\midrule
			6'b000000   & R-Type        \\
			\midrule
			6'b000001   & bltz/bgez     \\
			\midrule
			6'b001101   & ori           \\
			\midrule
			6'b100011   & lw            \\
			\midrule
			6'b101011   & sw            \\
			\midrule
			6'b000100   & beq           \\
			\midrule
			6'b001111   & lui           \\
			\midrule
			6'b000010   & j             \\
			\midrule
			6'b000011   & jal           \\
			\midrule
		\end{tabular}
	\end{threeparttable}
\end{table}
\begin{table}[H]
	\centering
	\begin{threeparttable}
		\caption{Funct译码}
		\begin{tabular}{cc}
			\toprule
			\rowcolor{mypink}
			\textbf{Funct} & \textbf{Inst} \\
			\midrule
			6'b100001      & addu          \\
			\midrule
			6'b100011      & subu          \\
			\midrule
			6'b000000      & sll           \\
			\midrule
			6'b001000      & jr            \\
			\midrule
			6'b001001      & jalr          \\
			\midrule
		\end{tabular}
	\end{threeparttable}
\end{table}

\begin{table}[H]
	\centering
	\begin{threeparttable}
		\caption{指令-输出}
		\begin{tabular}{|c|c|c|c|c|c|c|c|c|c|}
			\hline
			\rowcolor{mypink}
			\diagbox{\textbf{Inst}}{\textbf{Output}} & \textbf{MemtoReg} & \textbf{MemWrite} & \textbf{Branch} & \textbf{BranchOp} & \textbf{ALUCtrl} & \textbf{ALUASrc} & \textbf{ALUSrc} & \textbf{RegDst} & \textbf{RegWrite} \\
			\hline
			addu                                     & 0                 & 0                 & 0               & 0                 & 4'b0000          & 0                & 0               & 1               & 1                 \\
			\hline
			subu                                     & 0                 & 0                 & 0               & 0                 & 4'b0001          & 0                & 0               & 1               & 1                 \\
			\hline
			ori                                      & 0                 & 0                 & 0               & 0                 & 4'b0011          & 0                & 1               & 0               & 1                 \\
			\hline
			lw                                       & 1                 & 0                 & 0               & 0                 & 4'b0000          & 0                & 1               & 0               & 1                 \\
			\hline
			sw                                       & 0                 & 1                 & 0               & 0                 & 4'b0000          & 0                & 1               & 0               & 0                 \\
			\hline
			beq                                      & 0                 & 0                 & 1               & 4'b0000           & 4'b0001          & 0                & 0               & 0               & 0                 \\
			\hline
			lui                                      & 0                 & 0                 & 0               & 0                 & 4'b0111          & 0                & 1               & 0               & 1                 \\
			\hline
			sll                                      & 0                 & 0                 & 0               & 0                 & 4'b0100          & 1                & 0               & 1               & 1                 \\
			\hline
			sllv                                     & 0                 & 0                 & 0               & 0                 & 4'b0100          & 0                & 0               & 1               & 1                 \\
			\hline
			sra                                      & 0                 & 0                 & 0               & 0                 & 4'b0110          & 1                & 0               & 1               & 1                 \\
			\hline
			srav                                     & 0                 & 0                 & 0               & 0                 & 4'b0110          & 0                & 0               & 1               & 1                 \\
			\hline
			srl                                      & 0                 & 0                 & 0               & 0                 & 4'b0101          & 1                & 0               & 1               & 1                 \\
			\hline
			srlv                                     & 0                 & 0                 & 0               & 0                 & 4'b0101          & 0                & 0               & 1               & 1                 \\
			\hline
			j                                        & 0                 & 0                 & 0               & 0                 & 0                & 0                & 0               & 0               & 0                 \\
			\hline
			jr                                       & 0                 & 0                 & 0               & 0                 & 0                & 0                & 0               & 1               & 0                 \\
			\hline
			jal                                      & 0                 & 0                 & 0               & 0                 & 0                & 0                & 0               & 0               & 1                 \\
			\hline
			jalr                                     & 0                 & 0                 & 0               & 0                 & 0                & 0                & 0               & 1               & 1                 \\
			\hline
		\end{tabular}
	\end{threeparttable}
\end{table}
\begin{table}[H]
	\centering
	\begin{threeparttable}
		\caption{续上表}
		\begin{tabular}{|c|c|c|c|c|}
			\hline
			\rowcolor{mypink}
			\diagbox{\textbf{Inst}}{\textbf{Output}} & \textbf{Extend} & \textbf{Jump} & \textbf{Jump\_R} & \textbf{Link} \\
			\hline
			addu                                     & 0               & 0             & 0                & 0             \\
			\hline
			subu                                     & 0               & 0             & 0                & 0             \\
			\hline
			ori                                      & 0               & 0             & 0                & 0             \\
			\hline
			lw                                       & 1               & 0             & 0                & 0             \\
			\hline
			sw                                       & 1               & 0             & 0                & 0             \\
			\hline
			beq                                      & 1               & 0             & 0                & 0             \\
			\hline
			lui                                      & 1               & 0             & 0                & 0             \\
			\hline
			sll                                      & 0               & 0             & 0                & 0             \\
			\hline
			sllv                                     & 0               & 0             & 0                & 0             \\
			\hline
			sra                                      & 0               & 0             & 0                & 0             \\
			\hline
			srav                                     & 0               & 0             & 0                & 0             \\
			\hline
			srl                                      & 0               & 0             & 0                & 0             \\
			\hline
			srlv                                     & 0               & 0             & 0                & 0             \\
			\hline
			j                                        & 0               & 1             & 0                & 0             \\
			\hline
			jr                                       & 0               & 0             & 1                & 0             \\
			\hline
			jal                                      & 0               & 1             & 0                & 1             \\
			\hline
			jalr                                     & 0               & 0             & 1                & 1             \\
			\hline
		\end{tabular}
	\end{threeparttable}
\end{table}
\begin{figure}[H]
	\centering
	\includegraphics[scale=0.5]{Controller.png}
	\caption{Controller实现图}
\end{figure}
\newpage
\section{Testbench}
\begin{CenteredBox}
	\begin{lstlisting}
		.text
main:
		lui $1,0x114			# 3c010114
		lui $2,0x8930			# 3c028930
		ori $t1,$t1,32			# 35290020
		ori $t2,$t2,4			# 354a0004
		ori $t3,$t3,1			# 356b0001
		ori $t4,$t4,1			# 358c0001
loop:
		beq $t0,$t1,loop_end		# 11090006
		sw $t3,0($s0)			# ae0b0000
		lw $t3,0($s0)			# 8e0b0000
		addu $t3,$t3,$t4		# 016c5821
		addu $s0,$s0,$t2		# 020a8021
		addu $t0,$t0,$t4		# 010c4021
		beq $0,$0,loop			# 1000fff9
loop_end:
\end{lstlisting}
\end{CenteredBox}
\vspace{2ex}
\indent \textbf{期望状态}:\$1高位为0x114,\$2高位为0x8930,在DM中,从0x00到0x1f,被填入0x01到0x20。
\newpage
\section{思考题}
\subsection{若PC(程序计数器)位数为30位,试分析其与32位PC的优劣。}
由于目前MIPS的指令是定长的,都是按字对齐的,30位的PC也能起到与32位的PC相同的作用,且在跳转时不需要对偏移量进行移位。但是30位的PC无法支持变长的指令,不利于指令的扩展。
\subsection{现在我们的模块中 IM使用ROM, DM使用RAM, GRF使用寄存器,这种做法合理吗? 请给出分析,若有改进意见也请一并给出。}
目前是合理的:程序是较短小且一次性的,不需要更改,ROM是合适的;DM需要大量的空间,且需要是可读可写的,RAM是合适的;GRF需要高速读写,不需要太大的空间,也不需要扩展,寄存器的合适的。\\
\indent 当然,对于完整的CPU而言,IM应是可读写的,其数据来源可以是DM,但此时需要有一个bootloader才能成功启动。
\subsection{结合上文给出的样例真值表,给出RegDst, ALUSrc, MemtoReg,RegWrite, nPC\_Sel, ExtOp与op和func有关的布尔表达式(表达式中只能使用“与、或、非”3 种基本逻辑运算。)}
\begin{equation*}
	\begin{aligned}
		RegDst   & =\overline{op_5}\,\overline{op_4}\,\overline{op_3}\,\overline{op_2}\,\overline{op_1}\,\overline{op_0}\,\overline{f_5}f_4f_3f_2\overline{f_0}                                                                                                                         \\
		ALUSrc   & =\overline{op_5}\,\overline{op_4}op_3op_2\overline{op_1}op_0+op_5\overline{op_4}\,\overline{op_2}op_1op_0                                                                                                                                                            \\
		MemtoReg & =op_5\overline{op_4}\,\overline{op_3}\,\overline{op_2}op_1op_0                                                                                                                                                                                                       \\
		RegWrite & =\overline{op_5}\,\overline{op_4}\,\overline{op_3}\,\overline{op_2}\,\overline{op_1}\,\overline{op_0}\,\overline{f_5}f_4f_3f_2\overline{f_0}+\overline{op_5}\,\overline{op_4}op_3op_2\overline{op_1}op_0+op_5\overline{op_4}\,\overline{op_2}\overline{op_3}op_1op_0 \\
		nPC\_Sel & =\overline{op_5}\,\overline{op_4}\,\overline{op_3}op_2\overline{op_1}\,\overline{op_0}                                                                                                                                                                               \\
		ExtOp    & =op_5\overline{op_4}\,\overline{op_2}op_1op_0                                                                                                                                                                                                                        \\
	\end{aligned}
\end{equation*}
\subsection{充分利用真值表中的 X 可以将以上控制信号化简为最简单的表达式, 请给出化简后的形式。}
\begin{equation*}
	\begin{aligned}
		RegDst   & =\overline{op_5}\,\overline{op_4}\,\overline{op_3}\,\overline{op_2}\,\overline{op_1}\,\overline{op_0}\,\overline{f_5}f_4f_3f_2\overline{f_0}                                                                                                                         \\
		ALUSrc   & =\overline{op_5}\,\overline{op_4}op_3op_2\overline{op_1}op_0+op_5\overline{op_4}\,\overline{op_2}op_1op_0                                                                                                                                                            \\
		MemtoReg & =op_5\overline{op_4}\,\overline{op_2}op_1op_0                                                                                                                                                                                                                        \\
		RegWrite & =\overline{op_5}\,\overline{op_4}\,\overline{op_3}\,\overline{op_2}\,\overline{op_1}\,\overline{op_0}\,\overline{f_5}f_4f_3f_2\overline{f_0}+\overline{op_5}\,\overline{op_4}op_3op_2\overline{op_1}op_0+op_5\overline{op_4}\,\overline{op_2}\overline{op_3}op_1op_0 \\
		nPC\_Sel & =\overline{op_5}\,\overline{op_4}\,\overline{op_3}op_2\overline{op_1}\,\overline{op_0}                                                                                                                                                                               \\
		ExtOp    & =op_5\overline{op_4}\,\overline{op_2}op_1op_0
	\end{aligned}
\end{equation*}
\subsection{事实上,实现nop空指令,我们并不需要将它加入控制信号真值表,为什么?请给出你的理由。}
nop即空指令,仅用于占位,无实际作用,对于控制信号没有影响。
\subsection{前文提到,“可能需要手工修改指令码中的数据偏移”,但实际上只需再增加一个 DM片选信号,就可以解决这个问题。请阅读相关资料并设计一个 DM 改造方案使得无需手工修改数据偏移。}
改造方案大致示意图如下,将DM地址输入(图中输入信号的位数仅为示意)的最高位作为片选信号,最高位为零时是代码段,为一时是数据段。
\begin{figure}[H]
	\centering
	\includegraphics[scale=0.8]{Answer_DM.png}
	\caption{DM改造方案(使用Logisim析出)}
\end{figure}
\subsection{除了编写程序进行测试外,还有一种验证CPU设计正确性的办法——形式验证。 形式验证的含义是根据某个或某些形式规范或属性,使用数学的方法证明其正确性或非正确性。请搜索“形式验证(Formal Verification)"了解相关内容后,简要阐述相比与测试,形式验证的优劣。}
优点:
\begin{itemize}
	\item 能覆盖尽可能多的情况
	\item 能够很快地给出目标状态所需的条件
	\item 无需进行软件仿真或物理仿真,时间短
	\item 适用于多个模块层级
\end{itemize}
\indent \indent 缺点:
\begin{itemize}
	\item 对于大规模、状态复杂的电路系统,难以完成验证
	\item 仅在逻辑实现层面进行验证,无法验证电路实际情况的可靠性
	\item 仅能覆盖能够预计到的问题,无法真正地杜绝问题
\end{itemize}
\end{document}