\documentclass[UTF8]{ctexart}
\usepackage{amsmath}
\usepackage{float}
\usepackage{hyperref}
\usepackage{indentfirst}
\usepackage{cite}
\usepackage{multicol}
\usepackage{geometry}
\usepackage{txfonts}
\usepackage{extarrows}
\usepackage{graphicx}
\usepackage{booktabs}
\usepackage{diagbox}
\usepackage{threeparttable}
\usepackage{caption}
\usepackage[usenames,dvipsnames]{color}
\usepackage{colortbl}
\usepackage{fontspec}
\usepackage{listings}
\usepackage{mips}
\usepackage{fancybox}
\usepackage[super,square]{natbib}
\usepackage{multirow}

\pagestyle{myheadings}

\makeatletter
\newenvironment{CenteredBox}{% 
\begin{Sbox}}{% Save the content in a box
\end{Sbox}\centerline{\parbox{\wd\@Sbox}{\TheSbox}}}% And output it centered
\makeatother

\definecolor{vgreen}{RGB}{104,180,104}
\definecolor{vblue}{RGB}{49,49,255}
\definecolor{vorange}{RGB}{255,143,102}
\definecolor{dkgreen}{rgb}{0,0.6,0}
\definecolor{gray}{rgb}{0.5,0.5,0.5}
\definecolor{mauve}{rgb}{0.58,0,0.82}
\definecolor{lstrule}{RGB}{20,20,20}

\newfontfamily\courier{Courier New}
\lstdefinestyle{mips-style}
{ 
	language=[mips]Assembler,       % the language of the code
	numbers=left,
	numberstyle=\small\courier,
	basicstyle=\small\courier,
	frame=shadowbox,
	breaklines=true,
	rulesepcolor=\color{lstrule},
	xleftmargin=5em
}

\lstdefinestyle{verilog-style}
{
    language=Verilog,
    basicstyle=\small\courier,
    keywordstyle=\color{vblue},
    identifierstyle=\color{black},
    commentstyle=\color{vgreen},
    tabsize=8,
    moredelim=*[s][\colorIndex]{[}{]},
	literate=*{:}{:}1,
	frame=shadowbox,
	breaklines=true,
	rulesepcolor=\color{lstrule},
	xrightmargin=30em
}

\captionsetup[table]{singlelinecheck = false,justification=raggedleft}
\setmainfont{Times New Roman}
\definecolor{mypink}{rgb}{.99,.91,.95}
\geometry{left=0.3cm,right=2cm,top=2.5cm,bottom=2cm}
\setlength{\parindent}{2em}
\addtolength{\parskip}{-.4em}
\title{\textbf{P4设计文档}}
\author{伦泽标}
\date{\today}
\begin{document}


\null
\nointerlineskip
\vfill
\let\snewpage \newpage
\let\newpage \relax
\maketitle
\let \newpage \snewpage
\vfill
\break
\section{Instruction Set}
\begin{table}[H]
	\begin{threeparttable}
		\caption{现已支持的指令}
		\begin{tabular}{cccl}
			\toprule
			\rowcolor{mypink}
			\textbf{Inst} & \textbf{Part}                            & \textbf{Value}                & \textbf{Comment}                                                        \\
			\midrule
			addu          & {Op[6],rs[5],rt[5],rd[5],00000,Funct[6]} & Op=6'b000000, Funct=6b'100001 & rd$\leftarrow$rs+rt without overflow                                    \\
			\midrule
			addiu         & {Op[6],rs[5],rt[5],imm[16]}              & Op=6'b001001                  & rt$\leftarrow$rs+sign\_extend(imm) without overflow                     \\
			\midrule
			subu          & {Op[6],rs[5],rt[5],rd[5],00000,Funct[6]} & Op=6'b000000, Funct=6b'100011 & rd$\leftarrow$rs-rt without overflow                                    \\
			\midrule
			and           & {Op[6],rs[5],rt[5],rd[5],00000,Funct[6]} & Op=6'b000000, Funct=6'b100100 & rd$\leftarrow$rs \& rt                                                  \\
			\midrule
			andi          & {Op[6],rs[5],rt[5],imm[16]}              & Op=6'b001100                  & rt$leftarrow$rs \& zero\_extend(imm)                                    \\
			\midrule
			or            & {Op[6],rs[5],rt[5],rd[5],00000,Funct[6]} & Op=6'b000000, Funct=6'b100101 & rd$\leftarrow$rs|rt                                                     \\
			\midrule
			ori           & {Op[6],rs[5],rt[5],imm[16]}              & Op=6'b001101                  & rt$\leftarrow$rs | zero\_extend(imm)                                    \\
			\midrule
			lw            & {Op[6],base[5],rt[5],offset[16]}         & Op=6'b100011                  & rt$\leftarrow$memory[base+offset], sign\_extend(offset)                 \\
			\midrule
			sw            & {Op[6],base[5],rt[5],offset[16]}         & Op=6'b101011                  & memory[base+offset]$\leftarrow$rt, sign\_extend(offset)                 \\
			\midrule
			lh            & {Op[6],base[5],rt[5],imm[16]}            & Op=6'b100001                  & select with Addr[1], addr[0] should be 0                    \\
			\midrule
			lhu           & {Op[6],base[5],rt[5],imm[16]}            & Op=6'b100101                  & select with Addr[1], addr[0] should be 0, unsigned          \\
			\midrule
			sh            & {Op[6],base[5],rt[5],imm[16]}            & Op=6'b101001                  & select with Addr[1], addr[0] should be 0                    \\
			\midrule
			lb            & {Op[6],base[5],rt[5],imm[16]}            & Op=6'b100000                  & select with Addr[1:0]                                              \\
			\midrule
			lbu           & {Op[6],base[5],rt[5],imm[16]}            & Op=6'b100100                  & select with Addr[1:0], unsigned                                    \\
			\midrule
			sb            & {Op[6],base[5],rt[5],imm[16]}            & Op=6'b101000                  & select with Addr[1:0]                                              \\
			\midrule
			beq           & {Op[6],rs[5],rt[5],offset[16]}           & Op=6'b000100                  & rs==rt, sign\_extend(offset)<<2                                         \\
			\midrule
			bne           & {Op[6],rs[5],rt[5],offset[16]}           & Op=6'b000101                  & rs$\neq$rt,sign\_extend(offset)<<2                                      \\
			\midrule
			bgtz          & {Op[6],rs[5],00000,offset[16]}           & Op=6'b000111                  & rs>0, sign\_extend(offset)<<2                                           \\
			\midrule
			blez          & {Op[6],rs[5],00000,offset[16]}           & Op=6'b000110                  & rs$\leq$0, sign\_extend(offset)<<2                                      \\
			\midrule
			bgez          & {Op[6],rs[5],rt[5],offset[16]}           & Op=6'b000001, rt=5'b00001     & rs$\geq$0, sign\_extend(offset)<<2                                      \\
			\midrule
			bltz          & {Op[6],rs[5],rt[5],offset[16]}           & Op=6'b000001, rt=5'b00000     & rs<0, sign\_extend(offset)<<2                                           \\
			\midrule
			lui           & {Op[6],00000,rt[5],imm[16]}              & Op=6'b001111                  & rt$\leftarrow$imm<<16                                                   \\
			\midrule
			sll           & {Op[6],00000,rt[5],rd[5],s[5],Funct[6]}  & Op=6'b000000, Funct=6'b000000 & rd$\leftarrow$rt<<s                                                     \\
			\midrule
			sllv          & {Op[6],rs[5],rt[5],rd[5],0*5,Funct}      & Op=6'b000000, Funct=6'b000100 & rd$\leftarrow$rt<<rs[4:0]                                               \\
			\midrule
			sra           & {Op[6],0*5,rt[5],rd[5],s[5],Funct}       & Op=6'b000000, Funct=6'b000011 & rd$\leftarrow$rt>>s,arithmetic                                          \\
			\midrule
			srav          & {Op[6],rs[5],rt[5],rd[5],0*5,Funct[6]}   & Op=6'b000000, Funct=6'b000111 & rd$\leftarrow$rt>>rs[4:0],arithmetic                                   \\
			\midrule
			srl           & {Op[6],0*5,rt[5],rd[5],s[5],Funct[6]}    & Op=6'b000000, Funct=6'b000010 & rd$\leftarrow$rt>>s,logic                                              \\
			\midrule
			srlv          & {Op[6],rs[5],rt[5],rd[5],0*5,Funct[6]}   & Op=6'b000000, Funct=6'b000110 & rd$\leftarrow$rt>>rs[4:0],logic                                        \\
			\midrule
		\end{tabular}
	\end{threeparttable}
\end{table}
\begin{table}[H]
	\begin{threeparttable}
		\caption{续上表}
		\begin{tabular}{cccl}
			\toprule
			\rowcolor{mypink}
			\textbf{Inst} & \textbf{Part}                            & \textbf{Value}                & \textbf{Comment}                                                        \\
			\midrule
			slt           & {Op[6],rs[5],rt[5],rd[5],00000,Funct[6]} & Op=6'b000000, Funct=6'b101010 & rd$\leftarrow$rs<rt signed                                              \\
			\midrule
			slti          & {Op[6],rs[5],rt[5],imm[16]}              & Op=6'b001010                  & rt$leftarrow$rs<sign\_extend(imm) signed                                \\
			\midrule
			sltiu         & {Op[6],rs[5],rt[5],imm[16]}              & Op=6'b001011                  & rt$\leftarrow$(0||rs)<(0||sign\_extend)                                 \\
			\midrule
			sltu          & {Op[6],rs[5],rt[5],rd[5],00000,Funct[6]} & Op=6'b000000, Funct=6'b101011 & rd$\leftarrow$(0||rs)<(0||rt)                                           \\
			\midrule
			j             & {Op[6], j\_addr[26]}                     & Op=6'b000010                  & PC=nPC, PC $\leftarrow$ \{PC[31:28],j\_addr,00\}                        \\
			\midrule
			jr            & {Op[6],rs[5],0*10,0*5,Funct[6]}          & Op=6'b000000, Funct=6'b001000 & PC=nPC, PC$\leftarrow$rs                                                \\
			\midrule
			jal           & {Op[6],j\_addr[26]}                      & Op=6'b000011                  & PC=nPC, PC $\leftarrow$ \{PC[31:28],j\_addr,00\}, \$ra$\leftarrow$ PC+4 \\
			\midrule
			jalr          & {Op[6],rs[5],0*5,rd[5],0*5,Funct[6]}     & Op=6'b000000, Funct=6'b001001 & PC=nPC, PC$\leftarrow$ rs,rd$\leftarrow$PC+4                            \\
			\midrule
			nop           & 0                                        & 0                             & sll \$0,\$0,0                                                           \\
			\midrule
			xor           & {Op[6],rs[5],rt[5],rd[5],00000,Funct[6]} & Op=6'b000000, Funct=6'b100110 & rd$\leftarrow$rs$\wedge$rt                                                    \\
			\midrule
			xori          & {Op[6],rs[5],rt[5],imm[16]}              & Op=6'b001110                  & rt$\leftarrow$rs $\wedge$ zero\_extend(imm)                                   \\
			\midrule
		\end{tabular}
	\end{threeparttable}
\end{table}
\newpage
\section{Data Path}
\subsection{GRF}
\begin{table}[H]
	\centering
	\begin{threeparttable}
		\caption{GRF端口定义}
		\begin{tabular}{cccc}
			\toprule
			\rowcolor{mypink}
			\textbf{Port} & \textbf{Direction} & \textbf{Width} & \textbf{Description} \\
			\midrule
			clk           & Input              & 1              & 时钟信号             \\
			\midrule
			reset         & Input              & 1              & 复位信号             \\
			\midrule
			we            & Input              & 1              & 写使能信号           \\
			\midrule
			A1            & Input              & [4:0]          & 读寄存器地址1        \\
			\midrule
			A2            & Input              & [4:0]          & 读寄存器地址2        \\
			\midrule
			A3            & Input              & [4:0]          & 写寄存器地址         \\
			\midrule
			wd            & Input              & [31:0]         & 数据输入             \\
			\midrule
			r1            & Output             & [31:0]         & A1所指向寄存器的值   \\
			\midrule
			r2            & Output             & [31:0]         & A2所指向寄存器的值   \\
			\midrule
		\end{tabular}
	\end{threeparttable}
\end{table}
\begin{table}[H]
	\centering
	\begin{threeparttable}
		\caption{GRF功能定义}
		\begin{tabular}{cc}
			\toprule
			\rowcolor{mypink}
			\textbf{Port} & \textbf{Function}                                                      \\
			\midrule
			reset         & 同步复位                                                               \\
			\midrule
			we            & 当Clk上升沿来临,若reset为低电平且we为高电平,则将wd写入A3指向的寄存器 \\
			\midrule
		\end{tabular}
	\end{threeparttable}
\end{table}
\newpage
\subsection{ALU}
\begin{table}[H]
	\centering
	\begin{threeparttable}
		\caption{ALU端口定义}
		\begin{tabular}{cccc}
			\toprule
			\rowcolor{mypink}
			\textbf{Port} & \textbf{Direction} & \textbf{Width} & \textbf{Description} \\
			\midrule
			SrcA          & Input              & [31:0]         & ALU第一个操作数      \\
			\midrule
			SrcB          & Input              & [31:0]         & ALU第二个操作数      \\
			\midrule
			ALUCtrl       & Input              & [3:0]          & ALU功能控制          \\
			\midrule
			ALURes        & Output             & [31:0]         & ALU计算结果          \\
			\midrule
			Zero          & Output             & 1              & ALU计算结果是否为零  \\
			\midrule
		\end{tabular}
	\end{threeparttable}
\end{table}
\begin{table}[H]
	\centering
	\begin{threeparttable}
		\caption{ALU功能定义}
		\begin{tabular}{cc}
			\toprule
			\rowcolor{mypink}
			\textbf{ALUCtrl} & \textbf{Function}          \\
			\midrule
			4'b0000          & 加法运算                   \\
			\midrule
			4'b0001          & 减法运算                   \\
			\midrule
			4'b0010          & 与运算                     \\
			\midrule
			4'b0011          & 或运算                     \\
			\midrule
			4'b0100          & 异或运算                   \\
			\midrule
			4'b0101          & 或非运算                   \\
			\midrule
			4'b0110          & 左移运算(SrcB<<SrcA[4:0])  \\
			\midrule
			4'b0111          & 算术右移(SrcB>>>SrcA[4:0]) \\
			\midrule
			4'b1000          & 逻辑右移(SrcB>>SrcA[4:0])  \\
			\midrule
			4'b1001          & 加载到高位                 \\
			\midrule
			4'b1010          & 比较大小(SrcA>SrcB)有符号  \\
			\midrule
			4'b1011          & 比较大小(SrcA<SrcB)无符号  \\
			\midrule
		\end{tabular}
	\end{threeparttable}
\end{table}
\newpage
\subsection{EXT}
\begin{table}[H]
	\centering
	\begin{threeparttable}
		\caption{EXT端口定义}
		\begin{tabular}{cccc}
			\toprule
			\rowcolor{mypink}
			\textbf{Port} & \textbf{Direction} & \textbf{Width}     & \textbf{Description} \\
			\midrule
			in            & Input              & 不定长,默认[15:0] & 输入                 \\
			\midrule
			type          & Input              & [1:0]              & Extender功能控制     \\
			\midrule
			out           & Output             & 不定长,默认[31:0] & 扩展结果             \\
			\midrule
		\end{tabular}
	\end{threeparttable}
\end{table}
\begin{table}[H]
	\centering
	\begin{threeparttable}
		\caption{EXT功能定义}
		\begin{tabular}{cc}
			\toprule
			\rowcolor{mypink}
			\textbf{type} & \textbf{Function} \\
			\midrule
			2'b00         & 零扩展            \\
			\midrule
			2'b01         & 符号扩展          \\
			\midrule
		\end{tabular}
	\end{threeparttable}
\end{table}
\newpage
\subsection{DM}
\begin{table}[H]
	\centering
	\begin{threeparttable}
		\caption{DM端口定义}
		\begin{tabular}{cccc}
			\toprule
			\rowcolor{mypink}
			\textbf{Port} & \textbf{Direction} & \textbf{Width} & \textbf{Description}       \\
			\midrule
			clk           & Input              & 1              & 时钟信号                   \\
			\midrule
			reset         & Input              & 1              & 复位信号                   \\
			\midrule
			we            & Input              & 1              & 写入使能信号               \\
			\midrule
			type          & input              & [2:0]          & 数据类型选择               \\
			\midrule
			addr          & Input              & [31:0]         & 目标存储器地址             \\
			\midrule
			wd            & Input              & [31:0]         & 数据输入                   \\
			\midrule
			PC            & Input              & [31:0]         & 执行写入时的PC值(评测用) \\
			\midrule
			rd            & Output             & [31:0]         & 读取数据                   \\
			\midrule
		\end{tabular}
	\end{threeparttable}
\end{table}
\begin{table}[H]
	\centering
	\begin{threeparttable}
		\caption{DM功能定义}
		\begin{tabular}{cc}
			\toprule
			\rowcolor{mypink}
			\textbf{Port}       & \textbf{Function}                                                      \\
			\midrule
			reset               & 同步复位                                                               \\
			\midrule
			\multirow{5}*{type} & 3'b000: Word                                                           \\
			~                   & 3'b010: Half\_unsigned                                                 \\
			~                   & 3'b011: Half\_signed                                                   \\
			~                   & 3'b100: Byte\_unsigned                                                 \\
			~                   & 3'b101: Byte\_signed                                                   \\
			\midrule
			we                  & 当clk上升沿来临,若reset为低电平且we为高电平,则将wd写入addr指向的地址 \\
			\midrule
		\end{tabular}
	\end{threeparttable}
\end{table}
\newpage
\subsection{IFU}
\begin{table}[H]
	\centering
	\begin{threeparttable}
		\caption{PC端口定义}
		\begin{tabular}{cccc}
			\toprule
			\rowcolor{mypink}
			\textbf{Port} & \textbf{Direction} & \textbf{Width} & \textbf{Description}           \\
			\midrule
			clk           & Input              & 1              & 时钟信号                       \\
			\midrule
			reset         & Input              & 1              & 同步复位信号(32'h0000\_3000) \\
			\midrule
			nPC           & Input              & [31:0]         & 下一个PC值(从NPC获得)        \\
			\midrule
			PC            & Output             & [31:0]         & 当前PC值                       \\
			\midrule
		\end{tabular}
	\end{threeparttable}
\end{table}
\begin{table}[H]
	\centering
	\begin{threeparttable}
		\caption{PC功能定义}
		\begin{tabular}{cc}
			\toprule
			\rowcolor{mypink}
			\textbf{Port}     & \textbf{Function}                                      \\
			\midrule
			\multirow{2}*{PC} & 当clk上升沿来临,若reset为高电平则复位至32'h0000\_3000 \\
			~                 & 否则PC<=nPC                                            \\
			\midrule
		\end{tabular}
	\end{threeparttable}
\end{table}
\begin{table}[H]
	\centering
	\begin{threeparttable}
		\caption{NPC端口定义}
		\begin{tabular}{cccc}
			\toprule
			\rowcolor{mypink}
			\textbf{Port} & \textbf{Direction} & \textbf{Width} & \textbf{Description}     \\
			\midrule
			clk           & Input              & 1              & 时钟信号                 \\
			\midrule
			branch        & Input              & 1              & 是否b跳转                \\
			\midrule
			jump          & Input              & 1              & 是否普通j跳转            \\
			\midrule
			jump\_r       & Input              & 1              & 是否寄存器j跳转          \\
			\midrule
			PC            & Input              & [31:0]         & 当前PC值                 \\
			\midrule
			imm           & Input              & [31:0]         & 已移位并符号扩展的立即数 \\
			\midrule
			J\_Index      & Input              & [25:0]         & J跳转的偏移量            \\
			\midrule
			RD            & Input              & [31:0]         & JR跳转偏移量             \\
			\midrule
			PC4           & Output             & [31:0]         & PC+4值                   \\
			\midrule
			nPC           & Output             & [31:0]         & 下一个PC值               \\
			\midrule
		\end{tabular}
	\end{threeparttable}
\end{table}
\begin{table}[H]
	\centering
	\begin{threeparttable}
		\caption{IM端口定义}
		\begin{tabular}{cccc}
			\toprule
			\rowcolor{mypink}
			\textbf{Port} & \textbf{Direction} & \textbf{Width} & \textbf{Description} \\
			\midrule
			addr          & Input              & [31:0]         & 当前PC值             \\
			\midrule
			Inst          & Output             & [31:0]         & 当前指令             \\
			\midrule
		\end{tabular}
	\end{threeparttable}
\end{table}
\newpage
\subsection{Branch(用于控制b跳转)}
\begin{table}[H]
	\centering
	\begin{threeparttable}
		\caption{Branch端口定义}
		\begin{tabular}{cccc}
			\toprule
			\rowcolor{mypink}
			\textbf{Port} & \textbf{Direction} & \textbf{Width} & \textbf{Description}                  \\
			\midrule
			SrcA          & Input              & [31:0]         & 第一个操作数(与ALU第一个操作数相同) \\
			\midrule
			SrcB          & Input              & [31:0]         & 第二个操作数(与ALU第二个操作数相同) \\
			\midrule
			Branch        & Input              & 1              & 是否进行b跳转判定                     \\
			\midrule
			BranchOp      & Input              & [3:0]          & b跳转类型                             \\
			\midrule
			pc\_branch    & Output             & 1              & 是否进行b跳转                         \\
			\midrule
		\end{tabular}
	\end{threeparttable}
\end{table}
\begin{table}[H]
	\centering
	\begin{threeparttable}
		\caption{Branch功能定义}
		\begin{tabular}{cc}
			\toprule
			\rowcolor{mypink}
			\textbf{BranchOp} & \textbf{Function} \\
			\midrule
			4'b0000           & BEQ               \\
			\midrule
			4'b0001           & BNE               \\
			\midrule
			4'b0010           & BGTZ              \\
			\midrule
			4'b0011           & BLEZ              \\
			\midrule
			4'b0100           & BGEZ              \\
			\midrule
			4'b0101           & BLTZ              \\
			\midrule
		\end{tabular}
	\end{threeparttable}
\end{table}
\newpage
\section{Controller}
\begin{table}[H]
	\centering
	\begin{threeparttable}
		\caption{Controller端口定义}
		\begin{tabular}{cccc}
			\toprule
			\rowcolor{mypink}
			\textbf{Port} & \textbf{Direction} & \textbf{Width} & \textbf{Description}            \\
			\midrule
			inst          & Input              & [31:0]         & 当前指令                        \\
			\midrule
			MemtoReg      & Output             & 1              & 从DM读取数据到GRF               \\
			\midrule
			MemWrite      & Output             & 1              & 向DM写入来自GRF的数据           \\
			\midrule
			Branch        & Output             & 1              & b跳转                           \\
			\midrule
			BranchOp      & Output             & [3:0]          & b跳转类型                       \\
			\midrule
			ALUCtrl       & Output             & [3:0]          & ALU功能控制                     \\
			\midrule
			ALUASrc       & Output             & 1              & ALU.SrcA来源选择                \\
			\midrule
			ALUSrc        & Output             & 1              & ALU.SrcB来源选择                \\
			\midrule
			RegDst        & Output             & 1              & 目标寄存器选择([20:16],[15:11]) \\
			\midrule
			RegWrite      & Output             & 1              & 寄存器写使能                    \\
			\midrule
			Extend        & Output             & 1              & 立即数扩展类型                  \\
			\midrule
			Jump          & Output             & 1              & j跳转                           \\
			\midrule
			Jump\_R       & Output             & 1              & j跳转(r)                        \\
			\midrule
			Link          & Output             & 1              & j跳转(l)                        \\
			\midrule
			DataType      & Output             & [2:0]          & DM操作数据类型                  \\
			\midrule
		\end{tabular}
	\end{threeparttable}
\end{table}

\begin{table}[H]
	\centering
	\begin{threeparttable}
		\caption{指令-输出}
		\begin{tabular}{|c|c|c|c|c|c|c|c|c|c|}
			\hline
			\rowcolor{mypink}
			\diagbox{\textbf{Inst}}{\textbf{Output}} & \textbf{MemtoReg} & \textbf{MemWrite} & \textbf{Branch} & \textbf{BranchOp} & \textbf{ALUCtrl} & \textbf{ALUASrc} & \textbf{ALUSrc} & \textbf{RegDst} & \textbf{RegWrite} \\
			\hline
			addu                                     & 0                 & 0                 & 0               & 0                 & 4'b0000          & 0                & 0               & 1               & 1                 \\
			\hline
			andiu                                    & 0                 & 0                 & 0               & 0                 & 4'b0000          & 0                & 1               & 0               & 1                 \\
			\hline
			subu                                     & 0                 & 0                 & 0               & 0                 & 4'b0001          & 0                & 0               & 1               & 1                 \\
			\hline
			and                                      & 0                 & 0                 & 0               & 0                 & 4'b0010          & 0                & 0               & 1               & 1                 \\
			\hline
			andi                                     & 0                 & 0                 & 0               & 0                 & 4'b0010          & 0                & 1               & 0               & 1                 \\
			\hline
			or                                       & 0                 & 0                 & 0               & 0                 & 4'b0011          & 0                & 0               & 1               & 1                 \\
			\hline
			ori                                      & 0                 & 0                 & 0               & 0                 & 4'b0011          & 0                & 1               & 0               & 1                 \\
			\hline
			lw                                       & 1                 & 0                 & 0               & 0                 & 4'b0000          & 0                & 1               & 0               & 1                 \\
			\hline
			sw                                       & 0                 & 1                 & 0               & 0                 & 4'b0000          & 0                & 1               & 0               & 0                 \\
			\hline
			lh                                       & 1                 & 0                 & 0               & 0                 & 4'b0000          & 0                & 1               & 0               & 1                 \\
			\hline
			lhu                                      & 1                 & 0                 & 0               & 0                 & 4'b0000          & 0                & 1               & 0               & 1                 \\
			\hline
			sh                                       & 0                 & 1                 & 0               & 0                 & 4'b0000          & 0                & 1               & 0               & 0                 \\
			\hline
			lb                                       & 1                 & 0                 & 0               & 0                 & 4'b0000          & 0                & 1               & 0               & 1                 \\
			\hline
			lbu                                      & 1                 & 0                 & 0               & 0                 & 4'b0000          & 0                & 1               & 0               & 1                 \\
			\hline
			sb                                       & 0                 & 1                 & 0               & 0                 & 4'b0000          & 0                & 1               & 0               & 0                 \\
			\hline
			beq                                      & 0                 & 0                 & 1               & 4'b0000           & 4'b0001          & 0                & 0               & 0               & 0                 \\
			\hline
			bne                                      & 0                 & 0                 & 1               & 4'b0001           & 4'b0001          & 0                & 0               & 0               & 0                 \\
			\hline
			bgtz                                     & 0                 & 0                 & 1               & 4'b0010           & 4'b0001          & 0                & 0               & 0               & 0                 \\
			\hline
			blez                                     & 0                 & 0                 & 1               & 4'b0011           & 4'b0001          & 0                & 0               & 0               & 0                 \\
			\hline
			bgez                                     & 0                 & 0                 & 1               & 4'b0100           & 4'b0001          & 0                & 0               & 0               & 0                 \\
			\hline
			bltz                                     & 0                 & 0                 & 1               & 4'b0101           & 4'b0001          & 0                & 0               & 0               & 0                 \\
			\hline
			lui                                      & 0                 & 0                 & 0               & 0                 & 4'b1001          & 0                & 1               & 0               & 1                 \\
			\hline
			sll                                      & 0                 & 0                 & 0               & 0                 & 4'b0110          & 1                & 0               & 1               & 1                 \\
			\hline
			sllv                                     & 0                 & 0                 & 0               & 0                 & 4'b0110          & 0                & 0               & 1               & 1                 \\
			\hline
			sra                                      & 0                 & 0                 & 0               & 0                 & 4'b0111          & 1                & 0               & 1               & 1                 \\
			\hline
			srav                                     & 0                 & 0                 & 0               & 0                 & 4'b0111          & 0                & 0               & 1               & 1                 \\
			\hline
			srl                                      & 0                 & 0                 & 0               & 0                 & 4'b1000          & 1                & 0               & 1               & 1                 \\
			\hline
			srlv                                     & 0                 & 0                 & 0               & 0                 & 4'b1000          & 0                & 0               & 1               & 1                 \\
			\hline
			j                                        & 0                 & 0                 & 0               & 0                 & 0                & 0                & 0               & 0               & 0                 \\
			\hline
			jr                                       & 0                 & 0                 & 0               & 0                 & 0                & 0                & 0               & 1               & 0                 \\
			\hline
			jal                                      & 0                 & 0                 & 0               & 0                 & 0                & 0                & 0               & 0               & 1                 \\
			\hline
			jalr                                     & 0                 & 0                 & 0               & 0                 & 0                & 0                & 0               & 1               & 1                 \\
			\hline
			xor                                      & 0                 & 0                 & 0               & 0                 & 4'b0100          & 0                & 0               & 1               & 1                 \\
			\hline
			xori                                     & 0                 & 0                 & 0               & 0                 & 4'b0100          & 0                & 1               & 0               & 1                 \\
			\hline
			slt                                      & 0                 & 0                 & 0               & 0                 & 4'b1010          & 0                & 0               & 1               & 1                 \\
			\hline
			slti                                     & 0                 & 0                 & 0               & 0                 & 4'b1010          & 0                & 1               & 0               & 1                 \\
			\hline
			sltiu                                    & 0                 & 0                 & 0               & 0                 & 4'b1011          & 0                & 1               & 0               & 1                 \\
			\hline
			sltu                                     & 0                 & 0                 & 0               & 0                 & 4'b1011          & 0                & 0               & 1               & 1                 \\
			\hline
		\end{tabular}
	\end{threeparttable}
\end{table}
\begin{table}[H]
	\centering
	\begin{threeparttable}
		\caption{续上表}
		\begin{tabular}{|c|c|c|c|c|c|}
			\hline
			\rowcolor{mypink}
			\diagbox{\textbf{Inst}}{\textbf{Output}} & \textbf{Extend} & \textbf{Jump} & \textbf{Jump\_R} & \textbf{Link} & \textbf{DataType} \\
			\hline
			addu                                     & 0               & 0             & 0                & 0             & 0                 \\
			\hline
			addiu                                    & 1               & 0             & 0                & 0             & 0                 \\
			\hline
			subu                                     & 0               & 0             & 0                & 0             & 0                 \\
			\hline
			and                                      & 0               & 0             & 0                & 0             & 0                 \\
			\hline
			andi                                     & 0               & 0             & 0                & 0             & 0                 \\
			\hline
			or                                       & 0               & 0             & 0                & 0             & 0                 \\
			\hline
			ori                                      & 0               & 0             & 0                & 0             & 0                 \\
			\hline
			lw                                       & 1               & 0             & 0                & 0             & 3'b000            \\
			\hline
			sw                                       & 1               & 0             & 0                & 0             & 3'b000            \\
			\hline
			lh                                       & 1               & 0             & 0                & 0             & 3'b011            \\
			\hline
			lhu                                      & 1               & 0             & 0                & 0             & 3'b010            \\
			\hline
			sh                                       & 1               & 0             & 0                & 0             & 3'b010            \\
			\hline
			lb                                       & 1               & 0             & 0                & 0             & 3'b101            \\
			\hline
			lbu                                      & 1               & 0             & 0                & 0             & 3'b100            \\
			\hline
			sb                                       & 1               & 0             & 0                & 0             & 3'b100            \\
			\hline
			beq                                      & 1               & 0             & 0                & 0             & 0                 \\
			\hline
			bne                                      & 1               & 0             & 0                & 0             & 0                 \\
			\hline
			bgtz                                     & 1               & 0             & 0                & 0             & 0                 \\
			\hline
			blez                                     & 1               & 0             & 0                & 0             & 0                 \\
			\hline
			bgez                                     & 1               & 0             & 0                & 0             & 0                 \\
			\hline
			bltz                                     & 1               & 0             & 0                & 0             & 0                 \\
			\hline
			lui                                      & 0               & 0             & 0                & 0             & 0                 \\
			\hline
			sll                                      & 0               & 0             & 0                & 0             & 0                 \\
			\hline
			sllv                                     & 0               & 0             & 0                & 0             & 0                 \\
			\hline
			sra                                      & 0               & 0             & 0                & 0             & 0                 \\
			\hline
			srav                                     & 0               & 0             & 0                & 0             & 0                 \\
			\hline
			srl                                      & 0               & 0             & 0                & 0             & 0                 \\
			\hline
			srlv                                     & 0               & 0             & 0                & 0             & 0                 \\
			\hline
			j                                        & 0               & 1             & 0                & 0             & 0                 \\
			\hline
			jr                                       & 0               & 0             & 1                & 0             & 0                 \\
			\hline
			jal                                      & 0               & 1             & 0                & 1             & 0                 \\
			\hline
			jalr                                     & 0               & 0             & 1                & 1             & 0                 \\
			\hline
			xor                                      & 0               & 0             & 0                & 0             & 0                 \\
			\hline
			xori                                     & 0               & 0             & 0                & 0             & 0                 \\
			\hline
			slt                                      & 0               & 0             & 0                & 0             & 0                 \\
			\hline
			slti                                     & 1               & 0             & 0                & 0             & 0                 \\
			\hline
			sltiu                                    & 1               & 0             & 0                & 0             & 0                 \\
			\hline
			sltu                                     & 0               & 0             & 0                & 0             & 0                 \\
			\hline
		\end{tabular}
	\end{threeparttable}
\end{table}
\newpage
\section{Testbench}
\begin{center}
	\begin{lstlisting}[style={mips-style}]
				.text
		main:
				ori $1,$1,32 			# 34210020
				ori $2,$2,4 			# 34420004
				ori $3,$3,1 			# 34630001
				lui $4,0xffff			# 3c04ffff
				ori $7,0xffff			# 34e7ffff
				addu $4,$7,$4			# 00e42021
		
				jal loop			# 0c000c07
		loop:
				beq $6,$1,loop_end		# 10c10006
				sw $3,0($5) 			# aca30000
				lw $3,0($5) 			# 8ca30000
				subu $3,$3,$4 			# 00641823
				addu $5,$5,$2 			# 00a22821
				subu $6,$6,$4 			# 00c43023
				jr $ra				# 03e00008
		loop_end:
\end{lstlisting}
\end{center}
\vspace{2ex}
\indent \textbf{期望状态}:\$1高位为0x114,\$2高位为0x8930,在DM中,从0x00到0x1f,被填入0x01到0x20。
\newpage
\section{思考题}
\subsection{根据你的理解,在下面给出的DM的输入示例中,地址信号addr位数为什么是[11:2]而不是[9:0]?这个addr信号又是从哪里来的?}
\begin{figure}[H]
	\centering
	\includegraphics[scale=0.5]{./P4_L0_T2_5new.png}
\end{figure}
地址对齐,以字为单位进行寻址。\\
\indent addr来自ALU的计算结果。
\subsection{在相应的部件中,reset的优先级比其他控制信号(不包括clk信号)都要高,且相应的设计都是同步复位。清零信号reset是针对哪些部件进行清零复位操作?这些部件为什么需要清零?}
针对GRF,PC和DM。\\
\indent 这些部件都是有存储状态的,在开始运行前都需要将旧状态或无效状态复位为设定的初始状态。
\subsection{列举出用Verilog语言设计控制器的几种编码方式(至少三种),并给出代码示例。}
一、使用if-else:
\begin{lstlisting}[style={verilog-style}]
	begin	
		if(Op==`BEQ)
		begin
			Branch<=1;
		end
	end
\end{lstlisting}
\indent \indent 二、使用assign:
\begin{lstlisting}[style={verilog-style}]
	assign Branch=(Op==`BEQ)?1:0;
\end{lstlisting}
\indent\indent 三、使用case和宏定义:
\begin{lstlisting}[style={verilog-style}]
	case (Op)
	begin
		`BEQ:
			begin
				Branch<=1;
			end
	end
\end{lstlisting}
\subsection{根据你所列举的编码方式,说明他们的优缺点。}
一、使用if-else:
\begin{itemize}
	\item 结构整齐
	\item 繁复冗长
\end{itemize}
\indent\indent 二、使用assign:
\begin{itemize}
	\item 简短
	\item 以输出信号为主体,更清晰
	\item 复杂,难以分辨出其中的关系
\end{itemize}
\indent\indent 三、使用case和宏定义:
\begin{itemize}
	\item 结构整齐、清晰
	\item 繁复冗长
\end{itemize}
\subsection{请说明为什么在忽略溢出的前提下,addi与addiu是等价的,add与addu是等价的。}
addi/add与addiu/addu的计算步骤其实是相同的,都是无符号数加法,区别仅在于addi/add保留了加法运算的进位,并用此与结果最高位进行异或以判断是否溢出。
\subsection{根据自己的设计说明单周期处理器的优缺点。}
优点:
\begin{itemize}
	\item 结构简单清晰
	\item 不易发生数据竞争
\end{itemize}
\indent\indent 缺点:
\begin{itemize}
	\item 能够支持的指令少,无法支持需要多个周期的复杂指令
\end{itemize}
\subsection{简要说明jal、jr和堆栈的关系。}
在顺序指令执行的过程中,jal/jr与堆栈无直接关系,但是当涉及到了较复杂的函数调用关系时,为了能够顺利地沿着调用链回到调用前的状态,需要用堆栈把jal得到的ra存储起来,然后取出,用jr进行跳转。
\end{document}